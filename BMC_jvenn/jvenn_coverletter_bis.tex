%%%%%%%%%%%%%%%%%%%%%%%%%%%%%%%%%%%%%%%%%
% Long Lined Cover Letter
% LaTeX Template
% Version 1.0 (1/6/13)
%
% This template has been downloaded from:
% http://www.LaTeXTemplates.com
%
% Original author:
% Matthew J. Miller
% http://www.matthewjmiller.net/howtos/customized-cover-letter-scripts/
%
% License:
% CC BY-NC-SA 3.0 (http://creativecommons.org/licenses/by-nc-sa/3.0/)
%
%%%%%%%%%%%%%%%%%%%%%%%%%%%%%%%%%%%%%%%%%

%----------------------------------------------------------------------------------------
%	PACKAGES AND OTHER DOCUMENT CONFIGURATIONS
%----------------------------------------------------------------------------------------

\documentclass[10pt,stdletter,dateno,sigleft]{newlfm} % Extra options: 'sigleft' for a left-aligned signature, 'stdletternofrom' to remove the from address, 'letterpaper' for US letter paper - consult the newlfm class manual for more options

\usepackage{charter} % Use the Charter font for the document text

\newlfmP{sigsize=30pt} % Slightly decrease the height of the signature field

%----------------------------------------------------------------------------------------
%	YOUR NAME AND CONTACT INFORMATION
%----------------------------------------------------------------------------------------

\namefrom{Jerome Mariette} % Name

%----------------------------------------------------------------------------------------
%	ADDRESSEE AND GREETING/CLOSING
%----------------------------------------------------------------------------------------

\greetto{BMC bioinformatics editorial board,} % Greeting text
\closeline{Sincerely yours,} % Closing text

%----------------------------------------------------------------------------------------

\begin{document}
\begin{newlfm}

%----------------------------------------------------------------------------------------
%	LETTER CONTENT
%----------------------------------------------------------------------------------------

The present cover letter is pertaining to the paper "jvenn: an interactive venn
diagram viewer". This paper has already been submitted to BMC Bioinformatics 
under the ID 1449643681123201. After the review process, we were asked to
solve the concerns addressed. We made the requested modifications and we added 
new features to jvenn presented in the manuscript. Hereafter are our answers 
to the reviewers.



\textbf{-- Reviewer 1 --}

\textbf{1. The manuscript is sloppy / hand-wavy / not detailed enough in many
ways. A few examples are listed below. The authors need to fix these before I
can make a decision.}

\textbf{- URL running off the page 3}

The URL have been changed to fit the page size. 

\textbf{- run-on sentences}

The organisation of the sections and paragraphs of the text has been modified.
Many sentences have been rewritten.

\textbf{- what is a standard linux computer?}

The sentence ``Using the running version on a standard Linux computer (1 cpu,
4GB of RAM)'' has been added.

\textbf{- "whoever has some WEB programming skills" is a very vague way of
defining a user base}

The sentences have been modified as following: ``whoever has some JavaScript 
programming skills''

\textbf{- I recommend rewording the first paragraph; I had a hard time following
the logic in the way it is currently written. The point you are trying to make 
should be more clearly stated: diagrams are easy to read for up to four
categories but more features are needed for 5 or 6.}

The Background section has been fully rewritten.

\textbf{- how many sets do [4] and [5] allow for?}

They respectivly handle 3 and 4 lists. This has been added to the manuscript.


\textbf{- there are many weird or poorly designed sentence structures (e.g.:
"The library provides an option to de ne the data inputs: series.") and quite a 
few typos ("embeded", "overlights"?). I recommend the authors have the manuscript
checked by an editor or at least an English speaker.}

Many sentences have been rewritten as requested. 

\textbf{- the paper was submitted in the software category but at the top it
says research? I was a little confused by this.}

The section title of the manuscript has been changed for ``Software''.

\textbf{2. The manuscript is more or less a list of features. I feel there needs
to be a little more depth and insight in a scholarly article. For example, how
do  the features mentioned in paragraph two of the results and discussion
alleviate  the problem mentioned in the first paragraph? I have checked the two
papers  that cite the software (10,11) and both show only the venn diagram and
not  any of the other lists. How did the papers (10,11) use it and why couldn't 
they use another library? Perhaps a brief mention of other ways to
look/visualize this data might be worth it:}



\textbf{-- Reviewer 2 --}

\textbf{ -Major Compulsory Revisions
For the manuscript:
1. In the Background section, the authors list a number of libraries designed
for web applications such as BioVenn, venny, Canvasxpress, and the Google Chart 
API. However, there are number of other Venn diagram drawing libraries that are 
not directly embeddable in a webpage. Yet, the authors should note these
libraries as well. Two that come to mind are: The VennDiagram R package, which 
supports up to five lists. VENNTURE, which supports up to six lists Martin B, 
Chadwick W, Yi T, Park SS, Lu D, Ni B, Gadkaree S, Farhang K, Becker KG,
Maudsley  S. VENNTURE Novel Venn Diagram Investigational Tool for Multiple 
Pharmacological Dataset Analysis.}

These 2 citations and the description of the tools have been added in the
Background section.

\textbf{2. The authors claim that jvenn is the only library to accommodate up
to six input lists. VENNTURE (cited above) can produce Edwards-Venn
diagrams with up to six lists; thus their statement is incorrect.}

VENNTURE citation has been added to the manuscript. jvenn remains the only
tools handling 6 lists as inputs written to be included in Web applications.
Moreover it is the only software package offering classical Venn diagram displays
for 6 input lists.

\textbf{Also the authors point out that venny does not visualize the relative
areas of the intersecting sets; neither does jvenn, from what I can
tell. Is this a current limitation of jvenn or an explicit decision?}

This is right, all softwares displaying a relative areas Venn diagram handle up
to 3 lists. Beyond that, displaying the graphic is a complex problem.

\textbf{3. The jvenn library appears to be built upon the venny library;
however this distinction is unclear. In the implementation section,
the authors should more clearly emphasize their contribution to the
software, which appears to be modifying venny to handle more than 4
sets of items. A clear statement about the novel contributions (with
respect to the library) will strengthen the paper.}

jvenn is only using the venny algorithm to compute the lists intersection.
The sentence ``jvenn intersection computing algorithm is based on venny.
The algorithm has been wrapped in a jQuery plug-in,
extended with new features presented hereunder and the ability to use up to six
lists.'' has been added as introduction to the Implementation section.


\textbf{4. The authors should describe the three different data formats.
Table 1 shows good examples of the inputs, but "count lists" is
particularly unintuitive. This description could go in the
Implementation section, where the authors reference the table.}

A full explanation of the formats and a specific example of the "count lists"
has been added in the Implementation section.

\textbf{For the web application: 5. There is no help page on the web
application. The wiki page is quite useful, however. At the very least, there
should be a link to the wiki page to help answer user's questions. Adding a 
tutorial might increase usage as well.}

The full documentation is now online and a user manuel has been written to help
our Web application users.

\textbf{6. The web application was down for a time between 4/12 and 4/13,
which slowed the revision process. If this is a semi-regular
occurrence, I advise moving the application to a more reliable server.}

This was unfortunate, the server is reliable and monitored by our system
administrator team.

\textbf{7. The "on/off" switch buttons for the classic view and six lists are
quite confusing. I expected the diagram to change when I clicked
these. Perhaps a title or a box explaining their use would be better.}

We made modifications on this feature. The "on/off" switch buttons now
highlights the intersection figure.


\textbf{Minor Essential Revisions
Throughout: \newline
- WEB -> web or Web \newline
- javascript -> JavaScript \newline
- developper -> developer \newline
- Be consistent with six vs. 6 \newline
Title and Abstract: \newline
- venn should be capitalized in the title (though BMC Bioinformatics
might make everything lower-case - disregard if this is true). \newline
- embadable -> embeddable \newline
- allows [users] to control and customize all user interaction \newline
- awkward phrase: helping scientists to analyse their data \newline 
- The library package, [which comes] with [] full documentation \newline
- vizualization -> visualization \newline
Background: \newline
- awkward phrase: Biological projects are increasingly multiplexing
samples - what's a multiplexing sample? \newline
- availble -> available \newline
- These libraries generate [] graphical \newline
- cannot perform [] calculations \newline
- awkward phrase: permits to overload the callback function \newline}

All these spelling and sentence errors have been taken into account.

\textbf{Results and Discussion:\newline
- In general, there are many sentences in this section that are more
implementation details (such as the question mark, pop-ups, and
mouse-over). If possible, try to describe these items in the
Implementation section.}

All parameters description were moved to the Implementation as recommended.

\textbf{- Nowadays, this type of diagrams is able to present -> Nowadays,
these types of diagrams are able to present\newline
- "Nowadays" is incorrect - Venn diagrams have always been
able to visualize multiple lists.\newline
- diagrams are not proportional [to] the list counts\newline
- functionnalities -> functionalities\newline
- figures readibility -> figure's readability\newline
- on click function -> one-click function\newline
- awkward phrase: It allows users to check the list size homogeneity.\newline
- The second one[] displays the number\newline
- jvenn performances -> jvenn's performance\newline}

All these spelling and sentence errors have been taken into account.

\textbf{Conclusions and References:\newline
- The fact that jvenn is already embedded within four web applications
is impressive! I would add this fact in the Background section.}

Taken into account.

\textbf{- Venny is incorrectly cited - please provide the author and year}

The citation has been modified to add the author and year.

\textbf{- Some of the references have incorrect author names: last names are
followed by a,b,c, etc , presumably from institution references. Please fix.}

All the references were checked and standardized.

\textbf{Discretionary Revisions
For the manuscript:
- In the background, "allows three different data formats
(lists/intersection counts/count lists)" is misleading - looks like
two fractions and an integer. I'd suggest (list, intersection counts,
and count lists).
- The reference to jvenn in citation [11] is a dead link.}

Taken into account.

\textbf{- The authors should describe the difference between a classical and
Edwards-Venn diagrams. This description can be brief (1-2 sentences).}

A description of Edwards-Venn diagrams has been added to the manuscript. 

\textbf{- The jvenn url goes off the page under 'Availability and requirements'
For the web application:}

The URL have been changed to fit the page size. 

\textbf{- The text box under "click on a number to show the corresponding
elements" is writable, which was initially confusing.}

This has been changed, the textarea is set as readonly, users are no longer able 
to edit it but can still copy its content.

\textbf{- In the horizontal bar plot, labeling the numbers in the box and the
numbers on the x-axis would clarify the plot.}

The graphic title has been changed for ``Number of elements: specific(1) or
shared by 2, 3, \ldots lists''. This graphic is also explained in the
user manual section.


This paper is unpublished and has not been submitted for publication elsewhere. 
We would be grateful if you would consider it for publication in BMC 
bioinformatics.

%----------------------------------------------------------------------------------------

\end{newlfm}
\end{document}