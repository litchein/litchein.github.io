%% BioMed_Central_Tex_Template_v1.06
%%                                      %
%  bmc_article.tex            ver: 1.06 %
%                                       %

%%IMPORTANT: do not delete the first line of this template
%%It must be present to enable the BMC Submission system to
%%recognise this template!!

%%%%%%%%%%%%%%%%%%%%%%%%%%%%%%%%%%%%%%%%%
%%                                     %%
%%  LaTeX template for BioMed Central  %%
%%     journal article submissions     %%
%%                                     %%
%%          <8 June 2012>              %%
%%                                     %%
%%                                     %%
%%%%%%%%%%%%%%%%%%%%%%%%%%%%%%%%%%%%%%%%%


%%%%%%%%%%%%%%%%%%%%%%%%%%%%%%%%%%%%%%%%%%%%%%%%%%%%%%%%%%%%%%%%%%%%%
%%                                                                 %%
%% For instructions on how to fill out this Tex template           %%
%% document please refer to Readme.html and the instructions for   %%
%% authors page on the biomed central website                      %%
%% http://www.biomedcentral.com/info/authors/                      %%
%%                                                                 %%
%% Please do not use \input{...} to include other tex files.       %%
%% Submit your LaTeX manuscript as one .tex document.              %%
%%                                                                 %%
%% All additional figures and files should be attached             %%
%% separately and not embedded in the \TeX\ document itself.       %%
%%                                                                 %%
%% BioMed Central currently use the MikTex distribution of         %%
%% TeX for Windows) of TeX and LaTeX.  This is available from      %%
%% http://www.miktex.org                                           %%
%%                                                                 %%
%%%%%%%%%%%%%%%%%%%%%%%%%%%%%%%%%%%%%%%%%%%%%%%%%%%%%%%%%%%%%%%%%%%%%

%%% additional documentclass options:
%  [doublespacing]
%  [linenumbers]   - put the line numbers on margins

%%% loading packages, author definitions

%\documentclass[twocolumn]{bmcart}% uncomment this for twocolumn layout and comment line below
\documentclass{bmcart}

%%% Load packages
\usepackage{listings}
\usepackage{color}
\definecolor{gray}{rgb}{0.5,0.5,0.5}
\lstset{
  language=Java,
  showstringspaces=false,
  columns=flexible,
  basicstyle={\scriptsize \ttfamily},
  numbers=none,
  stringstyle=\color{gray},
  breaklines=true,
  breakatwhitespace=true
}
%\usepackage{amsthm,amsmath}
%\RequirePackage{natbib}
%\RequirePackage{hyperref}
\usepackage[utf8]{inputenc} %unicode support
%\usepackage[applemac]{inputenc} %applemac support if unicode package fails
%\usepackage[latin1]{inputenc} %UNIX support if unicode package fails


%%%%%%%%%%%%%%%%%%%%%%%%%%%%%%%%%%%%%%%%%%%%%%%%%
%%                                             %%
%%  If you wish to display your graphics for   %%
%%  your own use using includegraphic or       %%
%%  includegraphics, then comment out the      %%
%%  following two lines of code.               %%
%%  NB: These line *must* be included when     %%
%%  submitting to BMC.                         %%
%%  All figure files must be submitted as      %%
%%  separate graphics through the BMC          %%
%%  submission process, not included in the    %%
%%  submitted article.                         %%
%%                                             %%
%%%%%%%%%%%%%%%%%%%%%%%%%%%%%%%%%%%%%%%%%%%%%%%%%


\def\includegraphic{}
\def\includegraphics{}



%%% Put your definitions there:
\startlocaldefs
\endlocaldefs


%%% Begin ...
\begin{document}

%%% Start of article front matter
\begin{frontmatter}

\begin{fmbox}
\dochead{Research}

%%%%%%%%%%%%%%%%%%%%%%%%%%%%%%%%%%%%%%%%%%%%%%
%%                                          %%
%% Enter the title of your article here     %%
%%                                          %%
%%%%%%%%%%%%%%%%%%%%%%%%%%%%%%%%%%%%%%%%%%%%%%

\title{jvenn: an interactive venn diagram viewer.}

%%%%%%%%%%%%%%%%%%%%%%%%%%%%%%%%%%%%%%%%%%%%%%
%%                                          %%
%% Enter the authors here                   %%
%%                                          %%
%% Specify information, if available,       %%
%% in the form:                             %%
%%   <key>={<id1>,<id2>}                    %%
%%   <key>=                                 %%
%% Comment or delete the keys which are     %%
%% not used. Repeat \author command as much %%
%% as required.                             %%
%%                                          %%
%%%%%%%%%%%%%%%%%%%%%%%%%%%%%%%%%%%%%%%%%%%%%%

\author[
   addressref={aff2},                   % id's of addresses, e.g. {aff1,aff2}
   noteref={n1},                        % id's of article notes, if any
   email={Philippe.Bardou@toulouse.inra.fr}   % email address
]{\inits{PB}\fnm{Philippe} \snm{Bardou}}
\author[
   addressref={aff1},
   corref={aff1},                       % id of corresponding address, if any
   noteref={n1},                        % id's of article notes, if any
   email={Jerome.Mariette@toulouse.inra.fr}
]{\inits{JM}\fnm{J\'{e}r\^{o}me} \snm{Mariette}}
\author[
   addressref={aff1},
   email={Christophe.Djemiel@toulouse.inra.fr}
]{\inits{CD}\fnm{Christophe} \snm{Djemiel}}
\author[
   addressref={aff1,aff2},
   email={Christophe.Klopp@toulouse.inra.fr}
]{\inits{CK}\fnm{Christophe} \snm{Klopp}}

%%%%%%%%%%%%%%%%%%%%%%%%%%%%%%%%%%%%%%%%%%%%%%
%%                                          %%
%% Enter the authors' addresses here        %%
%%                                          %%
%% Repeat \address commands as much as      %%
%% required.                                %%
%%                                          %%
%%%%%%%%%%%%%%%%%%%%%%%%%%%%%%%%%%%%%%%%%%%%%%

\address[id=aff1]{%                           % unique id
  \orgname{Plate-forme bio-informatique Genotoul / MIA-T, INRA}, % university, etc
  \street{Borde Rouge},                     %
  \postcode{31326}                                % post or zip code
  \city{Castanet-Tolosan},                              % city
  \cny{France}                                    % country
}
\address[id=aff2]{%
  \orgname{Plate-forme SIGENAE / GenPhySE, INRA}, % university, etc
  \street{Borde Rouge},                     %
  \postcode{31326}                                % post or zip code
  \city{Castanet-Tolosan},                              % city
  \cny{France}                                    % country
}

%%%%%%%%%%%%%%%%%%%%%%%%%%%%%%%%%%%%%%%%%%%%%%
%%                                          %%
%% Enter short notes here                   %%
%%                                          %%
%% Short notes will be after addresses      %%
%% on first page.                           %%
%%                                          %%
%%%%%%%%%%%%%%%%%%%%%%%%%%%%%%%%%%%%%%%%%%%%%%

\begin{artnotes}
%\note{Sample of title note}     % note to the article
\note[id=n1]{Equal contributor} % note, connected to author
\end{artnotes}

\end{fmbox}% comment this for two column layout

%%%%%%%%%%%%%%%%%%%%%%%%%%%%%%%%%%%%%%%%%%%%%%
%%                                          %%
%% The Abstract begins here                 %%
%%                                          %%
%% Please refer to the Instructions for     %%
%% authors on http://www.biomedcentral.com  %%
%% and include the section headings         %%
%% accordingly for your article type.       %%
%%                                          %%
%%%%%%%%%%%%%%%%%%%%%%%%%%%%%%%%%%%%%%%%%%%%%%

\begin{abstractbox}

\begin{abstract} % abstract
\parttitle{Background} %if any
The amount of rich WEB applications allowing scientists to store, share and analyze data online is increasing. 
This enhances the need of embadable visualization tools. Scientists often produce lists of known identifiers 
corresponding to different experimental conditions. The Venn diagram is one of the most popular chart types 
used to present list comparison results.

\parttitle{Results} %if any
jvenn is a javascript library providing lists processing and Venn diagram displaying functions. It is the only library 
able to handle up to 6 input lists presenting results as classical or Edwards-Venn diagrams. Using it, developpers can 
easily embed Venn diagramm visualization features in WEB pages. jvenn is fully configurable and allows to control 
and customize all user interactions.

\parttitle{Conclusions} %if any
We introduce jvenn an open source component for WEB environments helping scientists to analyse their data. The 
library package, comming with a full documentation and some integration examples, is freely available at 
https://mulcyber.toulouse.inra.fr/plugins/mediawiki/wiki/venny/index.php/Accueil.


\end{abstract}

%%%%%%%%%%%%%%%%%%%%%%%%%%%%%%%%%%%%%%%%%%%%%%
%%                                          %%
%% The keywords begin here                  %%
%%                                          %%
%% Put each keyword in separate \kwd{}.     %%
%%                                          %%
%%%%%%%%%%%%%%%%%%%%%%%%%%%%%%%%%%%%%%%%%%%%%%

\begin{keyword}
\kwd{Venn}
\kwd{Edward-Venn}
\kwd{vizualisation}
\kwd{jquery}
\kwd{javascript}
\end{keyword}

% MSC classifications codes, if any
%\begin{keyword}[class=AMS]
%\kwd[Primary ]{}
%\kwd{}
%\kwd[; secondary ]{}
%\end{keyword}

\end{abstractbox}
%
%\end{fmbox}% uncomment this for twcolumn layout

\end{frontmatter}

%%%%%%%%%%%%%%%%%%%%%%%%%%%%%%%%%%%%%%%%%%%%%%
%%                                          %%
%% The Main Body begins here                %%
%%                                          %%
%% Please refer to the instructions for     %%
%% authors on:                              %%
%% http://www.biomedcentral.com/info/authors%%
%% and include the section headings         %%
%% accordingly for your article type.       %%
%%                                          %%
%% See the Results and Discussion section   %%
%% for details on how to create sub-sections%%
%%                                          %%
%% use \cite{...} to cite references        %%
%%  \cite{koon} and                         %%
%%  \cite{oreg,khar,zvai,xjon,schn,pond}    %%
%%  \nocite{smith,marg,hunn,advi,koha,mouse}%%
%%                                          %%
%%%%%%%%%%%%%%%%%%%%%%%%%%%%%%%%%%%%%%%%%%%%%%

%%%%%%%%%%%%%%%%%%%%%%%%% start of article main body
% <put your article body there>

%%%%%%%%%%%%%%%%
%% Background %%
%%

\section*{Background}

Biological projects are increasingly multiplexing samples to assess differences between conditions or individuals, 
thus, it is important to provide researchers with effective visualization tools to explore and extract 
relevant differences between these data sets. Data analysis often produces lists of biological identifiers 
(gene names, operational taxonomic unit, ...) which are then compared. List intersection 
results are commonly visualized using Venn diagrams \cite{Venn1880} presenting shared and unshared identifier 
counts, providing an insight on the similarities between the lists.

Venn diagrams are often used to present results on WEB pages. Thus, several Venn diagram applications are 
availble. BioVenn \cite{Hulsen2008} or venny \cite{venny} are WEB applications with identifier input text 
areas. Where the first one offers only a three circles area-proportional diagram, the second one offers a 
four lists diagram without area proportion respect. Canvasxpress \cite{canvasxpress} and the Google Chart 
API \cite{googleAPI}, meanwhile, are javascript libraries including Venn diagram features which can easily 
be embedded in any WEB page. These libraries generate the graphical output given the intersection counts but 
cannot perform the calculations on the lists.

We introduce jvenn a javascript library helping scientists to present their data, in the same spirit as already 
existing tools such as jbrowse \cite{Westesson01032013}, Cytoscape-Web \cite{Lopes2010}, and jHeatmap \cite{DeuPons2014}. 
jvenn handles up to 6 input lists, can display classical or Edwards-Venn \cite{Edwards2004} diagrams, can easily be 
embeded in a WEB page, allows three different data formats (lists/intersection counts/count lists), exports PNG files and permits 
to overload the callback function in order to control users interactions. jvenn has already been cited in several scientific 
publications \cite{Bianchia2013, Aravindraja2013}.


\section*{Implementation}

jvenn is a javascript library written as a jQuery plugin \cite{jquery}. It can be embeded by referencing the javascript file 
in an HTML page. For researchers who want to produce a Venn diagram from their identifier lists, jvenn is also available as a 
WEB application at http://bioinfo.genotoul.fr/index.php?id=116. The installation documentation is included in the software package 
which can be downloaded from https://mulcyber.toulouse.inra.fr/plugins/mediawiki/wiki/venny/index.php/Accueil.


\section*{Results and Discussion}

jvenn outputs a chart build from two to six identifier lists. Overlap counts are displayed and are clickable enabling to access
identifiers list of the intersection. In order to ease overlaps understanding, on mouse over, jvenn highlights the intersection count
and the corresponding conditions while bluring the others.

The library provides an option to define the data inputs: \textit{series}. It accepts three different input formats discribed in 
Table 1. In the case of list or count lists \textit{series} it will first execute a function to compute the overlaps between lists and
display the chart. In the case of intersection counts \textit{series}, the plugin will only display the graphical results. The resulting 
display is based on the javascript canvas object allowing to export the chart as a PNG file. This last feature can be disabled in order 
to hide the exporting button from the user by setting the \textit{exporting} option to \textit{false}.

jvenn handles up to six lists, which leads to display sixty three overlap areas. Displaying and interacting with such a chart
can be bulky and difficult, therefore jvenn proposes a switch button panel allowing to activate or disactivate lists (Fig. 1). The selected 
overlap count is then displayed and can be clicked. Moreover for a high readability of the diagram, when the intersection counts size exceeds
the allowed space, the value is substituted by a question mark. The real value is then poped-up when the user mouse is over. This behaviour 
can be disabled by setting the \textit{shortNumber} option to \textit{false}. jvenn also provides the Edwards-Venn display (Fig. 2) available 
by setting the \textit{displayType} option to \textit{edwards}. This display gives a different graphical representation of the lists which is 
clearer for 6 list diagrams. The WEB application developer can also overload the callback function defining the click on an overlap number. 
This can be done by defining the \textit{fnClickCallback} parameter. The overloading function has access to the \textit{this.listnames} and
\textit{this.list} variables allowing the developer to control the user interactions. This feature can be disabled by setting the 
\textit{disableClick} option to \textit{true}. To customize the diagram display, the developer can also settle the \textit{colors} option.

Having an overview of the input data and comparing multiple diagrams can be difficult when using a Venn vizualisation. Thus, in order to ease 
this step, jvenn provides two extra charts bellow the Venn diagram (Fig. 1). The first graph represents the input lists size histogram. This one
allows the user to check the homogeneity of its lists size. As example, In Fig. 1, both SRR068051 and SRR068053 lists have more identifiers than 
other ones. The second graph, displays the number of elements shared by one to six lists. This feature can be used to compare multiple Venn 
diagrams in order to asses the similarities between the input lists. Setting the \textit{displayStat} to true enables this feature.

As example, we produced two venn diagrams representing six samples SRR068049, SRR06805, SRR068051, SRR068052, SRR068053 and 
SRR068054 corresponding to sets of Operational Taxonomic Units observed under different conditions. Fig. 1 shows intersections
between six of them using the Venn diagram display. In Fig. 2, jvenn overlights the intersection between three samples out of 
six.


\section*{Conclusions}

jvenn is an easy-to-use library which generates Venn and Edwards-Venn diagrams from lists of identifiers or from
computed intersection counts. Its implementation as a javascript library allows whoever has some WEB programming skills to embed it 
within a WEB page without any dependancies. jvenn is already embeded within nG6 \cite{Mariette2012}, RNAbrowse 
\cite{Mariette} and WallProtDB \cite{SanClemente} WEB applications.

\section*{Availability and requirements}

jvenn is freely available under the GNU General Public License (GPL) and can be downloaded with an example and the full documentation
at https://mulcyber.toulouse.inra.fr/plugins/mediawiki/wiki/venny/index.php/Accueil webcite. A running version is accessible at 
http://bioinfo.genotoul.fr/index.php?id=116.

%%%%%%%%%%%%%%%%%%%%%%%%%%%%%%%%%%%%%%%%%%%%%%
%%                                          %%
%% Backmatter begins here                   %%
%%                                          %%
%%%%%%%%%%%%%%%%%%%%%%%%%%%%%%%%%%%%%%%%%%%%%%

\begin{backmatter}

\section*{Competing interests}
The authors declare that they have no competing interests.

\section*{Author's contributions}
JM conceived and designed the project. JM, PB, and CD implemented the project. CK evaluated software capabilities, and provided 
feedback on implementation. JM and CK wrote the manuscript. All authors read and approved the final manuscript.

\section*{Acknowledgements}
We would like to acknowledge all our users for providing us useful feedback on
the system and for pointing out features worth developing.

%%%%%%%%%%%%%%%%%%%%%%%%%%%%%%%%%%%%%%%%%%%%%%%%%%%%%%%%%%%%%
%%                  The Bibliography                       %%
%%                                                         %%
%%  Bmc_mathpys.bst  will be used to                       %%
%%  create a .BBL file for submission.                     %%
%%  After submission of the .TEX file,                     %%
%%  you will be prompted to submit your .BBL file.         %%
%%                                                         %%
%%                                                         %%
%%  Note that the displayed Bibliography will not          %%
%%  necessarily be rendered by Latex exactly as specified  %%
%%  in the online Instructions for Authors.                %%
%%                                                         %%
%%%%%%%%%%%%%%%%%%%%%%%%%%%%%%%%%%%%%%%%%%%%%%%%%%%%%%%%%%%%%

% if your bibliography is in bibtex format, use those commands:
\bibliographystyle{bmc-mathphys} % Style BST file
\bibliography{bmc_jvenn}      % Bibliography file (usually '*.bib' )

% or include bibliography directly:
% \begin{thebibliography}
% \bibitem{b1}
% \end{thebibliography}

%%%%%%%%%%%%%%%%%%%%%%%%%%%%%%%%%%%
%%                               %%
%% Figures                       %%
%%                               %%
%% NB: this is for captions and  %%
%% Titles. All graphics must be  %%
%% submitted separately and NOT  %%
%% included in the Tex document  %%
%%                               %%
%%%%%%%%%%%%%%%%%%%%%%%%%%%%%%%%%%%

%%
%% Do not use \listoffigures as most will included as separate files

\section*{Figures}
  \begin{figure}[h!]
  \caption{\csentence{A six lists classic Venn diagram.}
      This Venn diagram displays overlaps between six different biological samples. When 
      the user clicks on a figure, it calls the developper defined function. 
      The icon located on the top-right, allows users to download the diagram as a PNG 
      file. On the bottom-right of the chart, a switch button panel allowing to activate or 
      disactivate lists to access a specific intersection count. The charts showing the lists 
      size repartition and the number of common and specific elements are located underneath the
      diagram.}
      \end{figure}

\begin{figure}[h!]
  \caption{\csentence{A six lists Edwards-Venn diagram.}
      On mouse over a figure, the shape corresponding to the list involved in the intersection
      are highlighted and the other ones faded out. In this example, the user points the
      intersection between samples SRR068049, SRR068051 and SRR068052 which contains eight 
      different items.}
      \end{figure}

%%%%%%%%%%%%%%%%%%%%%%%%%%%%%%%%%%%
%%                               %%
%% Tables                        %%
%%                               %%
%%%%%%%%%%%%%%%%%%%%%%%%%%%%%%%%%%%

%% Use of \listoftables is discouraged.
%%
\section*{Tables}
\begin{table}[h!]
\caption{Available formats and example for the \textit{series} option.}
      \begin{tabular}{cccc}
        \hline
        format & example\\ \hline
        lists & 
\begin{lstlisting}
series: [{
	name: 'sample1',
	data: ["Otu1", "Otu2", "Otu3", "Otu4", "Otu5", "Otu6", "Otu7"]
}, {
	name: 'sample2',
	data: ["Otu1", "Otu2", "Otu5", "Otu7", "Otu8", "Otu9"]
}]
\end{lstlisting}\\ \hline
        intersection counts & 
\begin{lstlisting}
series: [{
	name: {A: 'sample 1', B: 'sample 2', C: 'sample 3'},
	data: {A: 340, B: 562, C: 620, AB: 639, AC: 456, BC: 915, ABC: 552}
}]
\end{lstlisting}\\ \hline
        count lists  &
\begin{lstlisting}
series: [{
	name: 'sample1',
	data: ["Otu1", "Otu2", "Otu3", "Otu4", "Otu5", "Otu6", "Otu7"],
	values: [5, 15, 250, 20, 23, 58, 89]
}, {
	name: 'sample2',
	data: ["Otu1", "Otu2", "Otu5", "Otu7", "Otu8", "Otu9"],
	values: [90, 300, 10, 2, 45, 9]
}]
\end{lstlisting}\\ \hline
      \end{tabular}
\end{table}

\end{backmatter}
\end{document}