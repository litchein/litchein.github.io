%% BioMed_Central_Tex_Template_v1.06
%%                                      %
%  bmc_article.tex            ver: 1.06 %
%                                       %

%%IMPORTANT: do not delete the first line of this template
%%It must be present to enable the BMC Submission system to
%%recognise this template!!

%%%%%%%%%%%%%%%%%%%%%%%%%%%%%%%%%%%%%%%%%
%%                                     %%
%%  LaTeX template for BioMed Central  %%
%%     journal article submissions     %%
%%                                     %%
%%          <8 June 2012>              %%
%%                                     %%
%%                                     %%
%%%%%%%%%%%%%%%%%%%%%%%%%%%%%%%%%%%%%%%%%


%%%%%%%%%%%%%%%%%%%%%%%%%%%%%%%%%%%%%%%%%%%%%%%%%%%%%%%%%%%%%%%%%%%%%
%%                                                                 %%
%% For instructions on how to fill out this Tex template           %%
%% document please refer to Readme.html and the instructions for   %%
%% authors page on the biomed central website                      %%
%% http://www.biomedcentral.com/info/authors/                      %%
%%                                                                 %%
%% Please do not use \input{...} to include other tex files.       %%
%% Submit your LaTeX manuscript as one .tex document.              %%
%%                                                                 %%
%% All additional figures and files should be attached             %%
%% separately and not embedded in the \TeX\ document itself.       %%
%%                                                                 %%
%% BioMed Central currently use the MikTex distribution of         %%
%% TeX for Windows) of TeX and LaTeX.  This is available from      %%
%% http://www.miktex.org                                           %%
%%                                                                 %%
%%%%%%%%%%%%%%%%%%%%%%%%%%%%%%%%%%%%%%%%%%%%%%%%%%%%%%%%%%%%%%%%%%%%%

%%% additional documentclass options:
%  [doublespacing]
%  [linenumbers]   - put the line numbers on margins

%%% loading packages, author definitions

%\documentclass[twocolumn]{bmcart}% uncomment this for twocolumn layout and comment line below
\documentclass{bmcart}

%%% Load packages
\usepackage{listings}
\usepackage{color}
\definecolor{gray}{rgb}{0.5,0.5,0.5}
\lstset{
  language=Java,
  showstringspaces=false,
  columns=flexible,
  basicstyle={\scriptsize \ttfamily},
  numbers=none,
  stringstyle=\color{gray},
  breaklines=true,
  breakatwhitespace=true
}
%\usepackage{amsthm,amsmath}
%\RequirePackage{natbib}
%\RequirePackage{hyperref}
\usepackage[utf8]{inputenc} %unicode support
%\usepackage[applemac]{inputenc} %applemac support if unicode package fails
%\usepackage[latin1]{inputenc} %UNIX support if unicode package fails


%%%%%%%%%%%%%%%%%%%%%%%%%%%%%%%%%%%%%%%%%%%%%%%%%
%%                                             %%
%%  If you wish to display your graphics for   %%
%%  your own use using includegraphic or       %%
%%  includegraphics, then comment out the      %%
%%  following two lines of code.               %%
%%  NB: These line *must* be included when     %%
%%  submitting to BMC.                         %%
%%  All figure files must be submitted as      %%
%%  separate graphics through the BMC          %%
%%  submission process, not included in the    %%
%%  submitted article.                         %%
%%                                             %%
%%%%%%%%%%%%%%%%%%%%%%%%%%%%%%%%%%%%%%%%%%%%%%%%%


\def\includegraphic{}
\def\includegraphics{}



%%% Put your definitions there:
\startlocaldefs
\endlocaldefs


%%% Begin ...
\begin{document}\sloppy

%%% Start of article front matter
\begin{frontmatter}

\begin{fmbox}
\dochead{Software}

%%%%%%%%%%%%%%%%%%%%%%%%%%%%%%%%%%%%%%%%%%%%%%
%%                                          %%
%% Enter the title of your article here     %%
%%                                          %%
%%%%%%%%%%%%%%%%%%%%%%%%%%%%%%%%%%%%%%%%%%%%%%

\title{jvenn: an interactive Venn diagram viewer.}

%%%%%%%%%%%%%%%%%%%%%%%%%%%%%%%%%%%%%%%%%%%%%%
%%                                          %%
%% Enter the authors here                   %%
%%                                          %%
%% Specify information, if available,       %%
%% in the form:                             %%
%%   <key>={<id1>,<id2>}                    %%
%%   <key>=                                 %%
%% Comment or delete the keys which are     %%
%% not used. Repeat \author command as much %%
%% as required.                             %%
%%                                          %%
%%%%%%%%%%%%%%%%%%%%%%%%%%%%%%%%%%%%%%%%%%%%%%

\author[
   addressref={aff2},                   % id's of addresses, e.g. {aff1,aff2}
   noteref={n1},                        % id's of article notes, if any
   email={Philippe.Bardou@toulouse.inra.fr}   % email address
]{\inits{PB}\fnm{Philippe} \snm{Bardou}}
\author[
   addressref={aff1},
   corref={aff1},                       % id of corresponding address, if any
   noteref={n1},                        % id's of article notes, if any
   email={Jerome.Mariette@toulouse.inra.fr}
]{\inits{JM}\fnm{J\'{e}r\^{o}me} \snm{Mariette}}
\author[
   addressref={aff1},
   email={Frederic.Escudie@toulouse.inra.fr}
]{\inits{FE}\fnm{Fr\'{e}d\'{e}ric} \snm{Escudi\'{e}}}
\author[
   addressref={aff1},
   email={Christophe.Djemiel@toulouse.inra.fr}
]{\inits{CD}\fnm{Christophe} \snm{Djemiel}}
\author[
   addressref={aff1,aff2},
   email={Christophe.Klopp@toulouse.inra.fr}
]{\inits{CK}\fnm{Christophe} \snm{Klopp}}

%%%%%%%%%%%%%%%%%%%%%%%%%%%%%%%%%%%%%%%%%%%%%%
%%                                          %%
%% Enter the authors' addresses here        %%
%%                                          %%
%% Repeat \address commands as much as      %%
%% required.                                %%
%%                                          %%
%%%%%%%%%%%%%%%%%%%%%%%%%%%%%%%%%%%%%%%%%%%%%%

\address[id=aff1]{%                           % unique id
  \orgname{Plate-forme bio-informatique Genotoul / MIA-T, INRA}, % university, etc
  \street{Borde Rouge},                     %
  \postcode{31326}                                % post or zip code
  \city{Castanet-Tolosan},                              % city
  \cny{France}                                    % country
}
\address[id=aff2]{%
  \orgname{Plate-forme SIGENAE / GenPhySE, INRA}, % university, etc
  \street{Borde Rouge},                     %
  \postcode{31326}                                % post or zip code
  \city{Castanet-Tolosan},                              % city
  \cny{France}                                    % country
}

%%%%%%%%%%%%%%%%%%%%%%%%%%%%%%%%%%%%%%%%%%%%%%
%%                                          %%
%% Enter short notes here                   %%
%%                                          %%
%% Short notes will be after addresses      %%
%% on first page.                           %%
%%                                          %%
%%%%%%%%%%%%%%%%%%%%%%%%%%%%%%%%%%%%%%%%%%%%%%

\begin{artnotes}
%\note{Sample of title note}     % note to the article
\note[id=n1]{Equal contributor} % note, connected to author
\end{artnotes}

\end{fmbox}% comment this for two column layout

%%%%%%%%%%%%%%%%%%%%%%%%%%%%%%%%%%%%%%%%%%%%%%
%%                                          %%
%% The Abstract begins here                 %%
%%                                          %%
%% Please refer to the Instructions for     %%
%% authors on http://www.biomedcentral.com  %%
%% and include the section headings         %%
%% accordingly for your article type.       %%
%%                                          %%
%%%%%%%%%%%%%%%%%%%%%%%%%%%%%%%%%%%%%%%%%%%%%%

\begin{abstractbox}

\begin{abstract} % abstract
\parttitle{Background} %if any
Venn diagrams are commonly used to display list comparison. In biology, they 
are widely used to show the differences between gene lists originating from 
different differential analyses, for instance. They thus allow the comparison 
between different experimental conditions or between different methods. 
However, when the number of input lists exceeds four, the diagram becomes 
difficult to read. Alternative layouts and dynamic display features can improve 
its use and its readability.

\parttitle{Results} %if any
jvenn is a new JavaScript library. It processes lists and produces Venn diagrams. 
It handles up to six input lists and presents results using classical or Edwards-Venn 
layouts. User interactions can be controlled and customized. Finally, jvenn can 
easily be embeded in a web page, allowing to have dynamic Venn diagrams.

\parttitle{Conclusions} %if any
jvenn is an open source component for web environments helping
scientists to analyze their data. The library package, which comes with full
documentation and an example, is freely available at
http://bioinfo.genotoul.fr/jvenn.


\end{abstract}

%%%%%%%%%%%%%%%%%%%%%%%%%%%%%%%%%%%%%%%%%%%%%%
%%                                          %%
%% The keywords begin here                  %%
%%                                          %%
%% Put each keyword in separate \kwd{}.     %%
%%                                          %%
%%%%%%%%%%%%%%%%%%%%%%%%%%%%%%%%%%%%%%%%%%%%%%

\begin{keyword}
\kwd{Venn}
\kwd{Edward-Venn}
\kwd{vizualisation}
\kwd{jquery}
\kwd{JavaScript}
\end{keyword}

% MSC classifications codes, if any
%\begin{keyword}[class=AMS]
%\kwd[Primary ]{}
%\kwd{}
%\kwd[; secondary ]{}
%\end{keyword}

\end{abstractbox}
%
%\end{fmbox}% uncomment this for twcolumn layout

\end{frontmatter}

%%%%%%%%%%%%%%%%%%%%%%%%%%%%%%%%%%%%%%%%%%%%%%
%%                                          %%
%% The Main Body begins here                %%
%%                                          %%
%% Please refer to the instructions for     %%
%% authors on:                              %%
%% http://www.biomedcentral.com/info/authors%%
%% and include the section headings         %%
%% accordingly for your article type.       %%
%%                                          %%
%% See the Results and Discussion section   %%
%% for details on how to create sub-sections%%
%%                                          %%
%% use \cite{...} to cite references        %%
%%  \cite{koon} and                         %%
%%  \cite{oreg,khar,zvai,xjon,schn,pond}    %%
%%  \nocite{smith,marg,hunn,advi,koha,mouse}%%
%%                                          %%
%%%%%%%%%%%%%%%%%%%%%%%%%%%%%%%%%%%%%%%%%%%%%%

%%%%%%%%%%%%%%%%%%%%%%%%% start of article main body
% <put your article body there>

%%%%%%%%%%%%%%%%
%% Background %%
%%

\section*{Background}

High-throughput biology has led to an increasing number of data, with more and 
more complex experimental designs. The analysis of these data often produces 
biological identifier lists, including gene names or operational
taxonomic units, obtained from different methods (for differential analysis) 
or from different experimental conditions. Venn diagrams \cite{Venn1880} are a 
common visualization chart, which allows to spot shared and unshared identifiers
providing an insight on lists similarities. 

In a Venn diagram, each list is presented by a transparent shape. Shape overlaps
contain the elements shared between lists or more often the corresponding counts.
In proportional Venn diagrams, the size of a shape is proportional to the 
number of elements of the corresponding list or of the corresponding lists
intersection. Venn diagrams with up to four lists are easy to read and
understand but Venn diagrams with more than four lists, are much harder to
interpret. To solve this problem, the Edwards-Venn \cite{Edwards2004}
representation introduces new shapes providing a clearer view, shown in the 
example of Fig.~\ref{fig::edwards}.

Many Venn diagram software packages are already available. The first six lines
of Table~\ref{table::features} present a subset of selected packages with their 
main features (maximum number of input lists, input data formats, Venn diagram 
layouts, application types and output formats). The table gives insight on 
several aspects of Venn diagram production and highlights that, up to now, no
web application handled up to six lists. VENNTURE \cite{Bronwen2012} is the only
application able to produce such diagrams but it only implements Edwards layout
and runs only under MS-Windows OS, producing static MS-PowerPoint and MS-Excel
files. Proportional Venn diagrams can only display a very limited number of lists,
three at maximum. Therefore, as jvenn was designed to display up to six lists, 
this one, does not implement proportional diagrams.

Hereafter we introduce jvenn, a JavaScript library, developed as a jQuery
plug-in \cite{jquery}, including many features easing diagram production and
enhancing their readability. In particular, jvenn can handle up to 6 lists, is 
a dynamic tool and implements both proportional and Edwards layouts. The 
library has already been used and cited in two scientific publications
\cite{Bianchia2013, Aravindraja2013}. It is already embedded in different Web
applications such as nG6 \cite{Mariette2012}, RNAbrowse \cite{Mariette} and
WallProtDB \cite{SanClemente}.


\section*{Implementation}

This section presents the main features of the jvenn library, including the 
kind of inputs it accepts, the different types of charts it displays, the types 
of the outputs and how it can be integrated in websites or directly used on our 
example web page.

\subsection*{Inputs}

The jvenn library accepts three different input formats : ``Lists'',
``Intersection counts'' and ``Count lists''. Examples are presented in 
Table~\ref{table::format}, where the different lists are ``sample1'' and
``sample2'', the elements of the different lists are given in the fields
``data''. For ``Intersection counts'', the lists are given a label (``A'' or 
``B'') which is used to make the correspondence between the list and its count.
Finally, ``Count lists'' provide a count number for each element of a list.
Hence, with ``Count lists'' the figures presented in the diagram correspond to
the sums of counts of all elements shared between lists. they can be
particularly useful to present OTU (Operational Taxonomic Unit) read counts. For
``Lists'' and ``Count lists'', jvenn computes the intersection counts and
displays the chart. For ``intersection counts'', the intersection counts is
provided by the user. 

\subsection*{Display features}

Venn diagrams are commonly used to present up to six lists but for six 
lists, the intersection areas obtained when using a proportional layout are 
often too small to
display the figures.

To display five or six lists diagrams, in a user-friendly manner, jvenn
implements several features (the user interface is provided in Fig.~\ref{fig::features}). 
First, the layout can be switched between the standard layout and the
Edwards-Venn layout (Fig.~\ref{fig::edwards}) which gives a clearer graphical
representation for six lists diagrams. To enhance the figure's readability for
the classical six lists Venn chart, some count values are not shown and some are
display outside the chart, using lines to line the count to its corresponding
area. However, this is still not enough to show all figures. Therefore, a switch
button panel (right side of Fig.~\ref{fig::features}) was added. It enables to
switch on and off the different lists and to display the corresponding 
intersection counts. 
When the number of characters of the intersection count exceeds the available
space to display it, the value is substituted by a question mark. When the mouse
is mouved over this question mark, the value pops-up. To emphasize the list
involved in an intersection area, jvenn highlights the intersection shapes when
mouse is moved over, fading the others out.

The extra charts presented under the Venn diagram ease the verification and
comparison of multiple lists. The list size graph allows users to check the
homogeneity of the input list sizes. The intersection size graph can be used to
compare the compactness of multiple Venn diagrams.

Scientists are usually interested in extracting identifier lists for some
intersections, therefore, jvenn implements a one-click function which retrieves
the names of the corresponding sets and the identifiers. To find an identifier,
one can use a dynamic search box. The shapes containing the matching 
identifiers are highlighted when using this tool.

\subsection*{Outputs}

jvenn display is based on a JavaScript canvas object that allows for PNG 
export. The intersection table can also be downloaded as a CSV file. This file
contains a header line with the diagram area labels and, in column, the
identifiers of the elements contained in the area.

\subsection*{Integration}

jvenn allows programmers having only moderate JavaScript experiences to embed 
Venn diagrams in a Web page without dependency. It has been designed following the
examples of jbrowse \cite{Westesson01032013}, Cytoscape-Web \cite{Lopes2010},
and jHeatmap \cite{DeuPons2014}.
The integration documentation is included in the software package which can be
downloaded from http://bioinfo.genotoul.fr/jvenn.

\subsection*{Web application}

jvenn can also be directly used as a Web application, which is available at 
http://bioinfo.genotoul.fr/jvenn/example.html (Fig.~\ref{fig::web}).
jvenn's Web application performances depend on the client browser. Using the 
current version on a standard Linux computer (one cpu, four GB of RAM), it
displays a six lists diagram of 10,000 identifiers in two seconds.


\section*{Results}

M.A. Dillies and colleagues \cite{Dillies2012} have compared seven RNA-Seq data
normalization methods and given a set of best practices to help biologists in their 
data processing. In table 2, they have shown the differences between methods
pair-wise. The raw data table provided by the team contains 5,277 lines and
seven columns. The columns correspond to  the different methods presented in the
'Differential expression analysis' section of the article. The data in the table
was filtered ($p < 0.05$) to produce the method specific gene name lists. Six
out of seven lists were selected for further processing ; Med was left out.
The lists were uploaded to the jvenn application and a Venn diagram was
produced. Using the layout selector the diagram was shown in Edwards Venn
format, in which all figures are accessible. This view presents all the lists
overlap between methods. Considering Fig. 2, the higher values are located in
central areas of the graph showing that the methods share large portions of gene
lists. The list of 484 genes shared by DESeq, TMM, UQ and FQ has been extracted
by clicking on the corresponding figure. Gene G002562 was sought using the
search box. It was found to be part of the five genes shared by FQ and UQ. 

The jvenn statistics show that the different methods produce gene lists with
very different sizes (minimum 417 - maximum 1,249) and that most of the genes are
shared between methods : 1,069 genes out of 1,347 shared by at least four methods.

The same analysis was performed with VENNTURE, the only other tool able to
generate a six list Edwards Venn diagram. First, the software package was
installed on a computer running under MS-Windows. Then, the six gene lists were loaded
in an MS-Excel spreadsheet. VENNTURE was run using the spreadsheet as input
generating a static MS-PowerPoint file containing the diagram and a MS-Excel
file with all the intersection contents. The 484 genes shared by DESeq,
TMM, UQ and FQ  were found in the intersection spreadsheet. The diagram did not
allow to search for gene G002562. It was found using MS-Excel text search in the 
intersection spreadsheet.

\section*{Discussion}

jvenn enables to compare up to six lists and updates the diagram automatically
when modifying the lists content. Compared to VENNTURE it does not need any
installation and gives access to a dynamic diagram providing simple functions to
extract gene lists and perform searches. 
jvenns' statistical charts give a simple and quick overview of the sizes of the
different lists and of their overlaps. It permits to compare different Venn diagrams.
These features are not available in the VENNTURE software package.

For biologists using different techniques in their experiment or in their statistical
analysis, jvenn enables to quickly extract the shared identifiers. Being found 
using different approaches these elements present a higher confidence level.

\section*{Conclusion}

Thanks to its numerous features, dynamic behavior and graphical layout quality, jvenn can be 
efficiently used in many cases to compare different sets of results and easily extract shared 
elements. Being a simple JavaScript plug-in allows to embed it in any Web
environment.


\section*{Availability and requirements}

jvenn is freely available under the GNU General Public License (GPL) and can be
downloaded with an example and the full documentation at
http://bioinfo.genotoul.fr/jvenn  website. A running version is accessible at
http://bioinfo.genotoul.fr/jvenn/example.html.

%%%%%%%%%%%%%%%%%%%%%%%%%%%%%%%%%%%%%%%%%%%%%%
%%                                          %%
%% Backmatter begins here                   %%
%%                                          %%
%%%%%%%%%%%%%%%%%%%%%%%%%%%%%%%%%%%%%%%%%%%%%%

\begin{backmatter}

\section*{Competing interests}
The authors declare that they have no competing interests.

\section*{Author's contributions}
JM conceived and designed the project. JM, PB, FE and CD implemented the project.
CK evaluated software capabilities, and provided feedback on implementation. JM
and CK wrote the manuscript. All authors read and approved the final manuscript.

\section*{Acknowledgements}
We would like to acknowledge all our users for providing us useful feedback on
the system and for pointing out features worth developing. We thank the
reviewers for their insightful and constructive comments. 

%%%%%%%%%%%%%%%%%%%%%%%%%%%%%%%%%%%%%%%%%%%%%%%%%%%%%%%%%%%%%
%%                  The Bibliography                       %%
%%                                                         %%
%%  Bmc_mathpys.bst  will be used to                       %%
%%  create a .BBL file for submission.                     %%
%%  After submission of the .TEX file,                     %%
%%  you will be prompted to submit your .BBL file.         %%
%%                                                         %%
%%                                                         %%
%%  Note that the displayed Bibliography will not          %%
%%  necessarily be rendered by Latex exactly as specified  %%
%%  in the online Instructions for Authors.                %%
%%                                                         %%
%%%%%%%%%%%%%%%%%%%%%%%%%%%%%%%%%%%%%%%%%%%%%%%%%%%%%%%%%%%%%

% if your bibliography is in bibtex format, use those commands:
\bibliographystyle{bmc-mathphys} % Style BST file
\bibliography{bmc_jvenn}      % Bibliography file (usually '*.bib' )

% or include bibliography directly:
% \begin{thebibliography}
% \bibitem{b1}
% \end{thebibliography}

%%%%%%%%%%%%%%%%%%%%%%%%%%%%%%%%%%%
%%                               %%
%% Figures                       %%
%%                               %%
%% NB: this is for captions and  %%
%% Titles. All graphics must be  %%
%% submitted separately and NOT  %%
%% included in the Tex document  %%
%%                               %%
%%%%%%%%%%%%%%%%%%%%%%%%%%%%%%%%%%%

%%
%% Do not use \listoffigures as most will included as separate files

\section*{Figures}
  \begin{figure}[h!]
  \caption{\csentence{A six lists classic Venn diagram.}
      This Venn diagram displays overlaps between six different biological
      samples. The icon, located on the top-right, allows users to download the
      diagram as a PNG file. The middle-right switch button
      panel allows to activate or dis-activate lists to access a specific
      intersection count. Charts showing the list size and intersection size 
      repartition located underneath the diagram.}
      \end{figure}

\begin{figure}[h!]
  \caption{\csentence{A six lists Edwards-Venn diagram.}
      On mouse over a figure, the shape corresponding to the lists involved in
      the intersection are highlighted and the other ones faded out. In
      this example, the user pointed the intersection between samples SRR068049,
      SRR068051 and SRR068052 which contains eight different items.}
      \end{figure}

%%%%%%%%%%%%%%%%%%%%%%%%%%%%%%%%%%%
%%                               %%
%% Tables                        %%
%%                               %%
%%%%%%%%%%%%%%%%%%%%%%%%%%%%%%%%%%%

%% Use of \listoftables is discouraged.
%%
\section*{Tables}

\begin{table}[h!]
\caption{Features of a subset of already available software packages, and jvenn.}
	\begin{tabular}{c|cccccc}
		Application & Maximum & Layouts & Application &
		Proportionality & Input data & Output\\ 
		& number of & & type & & formats & formats \\
		& input lists & & & & & \\ \hline
		
		VENNTURE \cite{Bronwen2012} &  6 & Edwards & Stand-alone & No & Lists &
		Powerpoint\\ 
		& & & & & & and Excel \\ \hline
		
		VennDiagram \cite{RVennDiagram} &  5 & Classical & R package & No & Lists
		& R object \\
		& & & & & & and TIFF \\ \hline
		
		BioVenn \cite{Hulsen2008} &  3 & Classical & Web application & Yes &
		Lists & SVG and PNG \\ \hline
		
		venny \cite{venny} &  4 & Classical & Web application & No &
		Lists & PNG \\ \hline
		 
		Canvasxpress \cite{canvasxpress} &  4 & Classical & JavaScript library &
		No & Intersection & JavaScript \\
		& & & & & counts & canvas \\ \hline
		
		Google & 3 & Classical & JavaScript library &
		Yes & Lists & PNG \\ 
		Chart API \cite{googleAPI} & & & & & & \\ \hline \hline
			
		jvenn & 6 & Classical & Web application & No & Lists,
		& Interactive 
		\\
		& & and Edwards & and JavaScript & & intersection & diagram, \\
		& & & library & & counts and & PNG and CSV \\
		& & & & & Count lists & 
	\end{tabular}
\end{table}


\begin{table}[h!]
\caption{Available formats and example for the \textit{series} option.}
      \begin{tabular}{cccc}
        \hline
        Format & Example\\ \hline
        Lists & 
\begin{lstlisting}
series: [{
	name: 'sample1',
	data: ["Otu1", "Otu2", "Otu3", "Otu4", "Otu5", "Otu6", "Otu7"]
}, {
	name: 'sample2',
	data: ["Otu1", "Otu2", "Otu5", "Otu7", "Otu8", "Otu9"]
}]
\end{lstlisting}\\ \hline
        Intersection counts & 
\begin{lstlisting}
series: [{
	name: {A: 'sample 1', B: 'sample 2'},
	data: {A: ["Otu3", "Otu4", "Otu6"], B: ["Otu8", "Otu9"], AB: ["Otu1",
	"Otu2", "Otu5", "Otu7"]} }],
	values: {A: 3, B: 2, AB: 4}
\end{lstlisting}\\ \hline
        Count lists  &
\begin{lstlisting}
series: [{
	name: 'sample1',
	data: ["Otu1", "Otu2", "Otu3", "Otu4", "Otu5", "Otu6", "Otu7"],
	values: [5, 15, 250, 20, 23, 58, 89]
}, {
	name: 'sample2',
	data: ["Otu1", "Otu2", "Otu5", "Otu7", "Otu8", "Otu9"],
	values: [90, 300, 10, 2, 45, 9]
}]
\end{lstlisting}\\ \hline
      \end{tabular}
\end{table}

\end{backmatter}
\end{document}