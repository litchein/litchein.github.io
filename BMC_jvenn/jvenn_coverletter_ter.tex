%%%%%%%%%%%%%%%%%%%%%%%%%%%%%%%%%%%%%%%%%
% Long Lined Cover Letter
% LaTeX Template
% Version 1.0 (1/6/13)
%
% This template has been downloaded from:
% http://www.LaTeXTemplates.com
%
% Original author:
% Matthew J. Miller
% http://www.matthewjmiller.net/howtos/customized-cover-letter-scripts/
%
% License:
% CC BY-NC-SA 3.0 (http://creativecommons.org/licenses/by-nc-sa/3.0/)
%
%%%%%%%%%%%%%%%%%%%%%%%%%%%%%%%%%%%%%%%%%

%----------------------------------------------------------------------------------------
%	PACKAGES AND OTHER DOCUMENT CONFIGURATIONS
%----------------------------------------------------------------------------------------

\documentclass[10pt,stdletter,dateno,sigleft]{newlfm} % Extra options: 'sigleft' for a left-aligned signature, 'stdletternofrom' to remove the from address, 'letterpaper' for US letter paper - consult the newlfm class manual for more options

\usepackage{charter} % Use the Charter font for the document text

\newlfmP{sigsize=30pt} % Slightly decrease the height of the signature field

%----------------------------------------------------------------------------------------
%	YOUR NAME AND CONTACT INFORMATION
%----------------------------------------------------------------------------------------

\namefrom{Jerome Mariette} % Name

%----------------------------------------------------------------------------------------
%	ADDRESSEE AND GREETING/CLOSING
%----------------------------------------------------------------------------------------

\greetto{BMC bioinformatics editorial board,} % Greeting text
\closeline{Sincerely yours,} % Closing text

%----------------------------------------------------------------------------------------

\begin{document}
\begin{newlfm}

%----------------------------------------------------------------------------------------
%	LETTER CONTENT
%----------------------------------------------------------------------------------------

The present cover letter is pertaining to the paper "jvenn: an interactive Venn
diagram viewer". Hereafter are our answers to the reviewers.


\textbf{-- REVIEWER 1 --}

\textbf{1. The authors do a comprehensive job of describing other Venn
diagram viewers. The bulleted list could be turned into a paragraph
summarizing the benefits and limitations. However, they fail to
clearly explain the benefits of jvenn that overcome limitations of
other methods. The table provides this information, but it should go
in the last paragraph of the Background.}

The bulleted list has been removed and replaced by a paragraph, concluded by a
sentence summarizing the benefits of jvenn.


\textbf{2. In the first round of revisions, Reviewer 1 noted that the
manuscript is written like a set of features, with relatively little
depth. The user-interaction and web developer details make parts of the paper
sound more like an API than an article. Explaining different variable names
in the paper such as shortNumber & fnClickCallback are not important; what's
important is that the library supports the intersection of large lists by
modifying the information displayed, and allows developers to customize the 
callback function for selecting an intersection.}

All parameters and options descriptions have been removed. All sentences
explaining the configuration have been deleted from the manuscript.


\textbf{3. The authors should carefully consider the organization of the
paper, namely, the Implementation, Results & Discussion, and
Conclusions. There are implementation details described in the
Results & Discussion section, which de-emphasizes the major points of
the section. Further, the Conclusions section is not concluding the
main points of the article; instead it introduces other minor details
of jvenn that have not be adequately described.}

The organisation of the paper has been modified. The section ``implementation''
is now described by five sub sections: ``Inputs'', ``Display features'',
``Outputs'', ``Integration'' and ``Web application''. The ``Results &
Discussions'' section has been splited into two distincts sections.


\textbf{4. Some sentences are still awkwardly-phrased; instead of run-on
sentences, there are many sentences that are too short / sentence
fragments. Some examples include:\newline
- "All are JSON objects."\newline
- "jvenn intersection computing algorithm is based on venny."\newline
- "This is true for up to four lists."\newline
- "The value pops up on mouse-over."}

Taken into account.


\textbf{- First paragraph of Background: "Venn diagram[s] with up to four lists
are easy to read and understand..."\newline
- Capitalization consistency for "Count" and "Lists" when describing
the inputs.\newline
- Second paragraph of Results & Discussion: "jvenn implements (Fig. 1)
several functionalities" - move figure reference to end of sentence.\newline
- Conclusion: "whoever has some JavaScript programming" is too
informal. Refer to comment from Reviewer 1 in first round of reviews\newline}

Taken into account. The sentence ``whoever has some JavaScript programming \ldots''
has been replaced by ``jvenn allows programmers having only moderate JavaScript
experiences to embed Venn diagrams in a web page without dependency.''


\textbf{-- REVIEWER 2 --}

\textbf{The paragraph that starts with "The Web application developer can de#ne
the diagram display by setting the display Type option to edwards or classic..." -
sounds like something that should go on a webpage for the project. You can
keep information about what the system is able to do, what features it has, but I
think you can skip on some of the low level details about how it is achieved.
Readers will want to know what can be done and that it can be done
in an easy way. State this and skip the details.}

All options and parameters descriptions have been removed from the manuscript.


\textbf{For me, a lot of the information in results and discussion sound like
features and very much like the kind of information you have in your Implementation section. I
encourage you to think hard about what the features are and what the results are
and separate them. Results are things that couldn't be done before and can be
done know with your system. There's a difference between
a feature (method to do something) and what it can accomplish (result). Here are
a few examples:
"The extra charts presented under the Venn diagram simplify veri#cation and
comparison of multiple diagrams." - this is the result (though you don't provide
any proof that they do simplify...)
"the display can be switched to Edwards-Venn" - method
"gives a clearer graphical representation for six list diagrams" - result.}

The paper has been fully re-organised and a complet example of jvenn usage has
been added in the ``Results'' section.


\textbf{"whoever has some" - this sounds very hand wavy! I know what you mean
but try to make it sound more appropriate for a paper. "programers with moderate
Javascript experience"?}

This sentence has been replaced by: ``jvenn allows programmers having only
moderate JavaScript experiences to embed Venn diagrams in a web page without
dependency.''


\textbf{fix: "Venn diagramS with up to four lists are easy to read and
understand,"}

Taken into account.


\textbf{in: "With “Count lists” the #gures presented in the diagram
correspond to the sums of counts of all identi#ers shared between lists. This has
been used in diversity studies to present OTU (Operational Taxonomic Unit) read
counts" -- does "with cl" mean "when count lists are used?". Also, "this" at the
beginning of the second sentence is a little ambiguous... qualify it
"It includes, as well, search and intersection identi#ers export functions." - a little
unclear what will be exported}

This section has been fully rewritten.


\textbf{I'm not really satisfied with how the authors addressed my comments
about this being a bag of features and lacking scientific depth. However, since
this is a software component, that's probably fine and I'll let this pass as long
as the authors put in the effort to make the writing more rigorous}

All features description have been removed and a usage example has been added in
the ``Results'' section. jvenn benefits are then presented in the ``Discussion''
section.


\textbf{-- EDITOR'S COMMENTS --}

\textbf{1. the Results section needs to contain an example case study for a
biological problem, preferably following this template:\newline
* a short description of the biological problem, \newline
* a brief description of how the problem was approached using jvenn, \newline
* any interesting/unusual findings facilitated by jvenn, \newline
* any jvenn feedback on specific features used in this case, \newline
* how long it took to solve the problem using jvenn, and \newline
* a short description of how the problem would have been solved otherwise and
how long it would have taken.\newline}

An example provided by M.A. Dillies and colleagues have been added to the paper,
all the analysis steps have been described and compared to VENNTURE the only
tool able to produce six lists Venn diagrams.

\textbf{In contrast, the current Results section is a list of jvenn
functionality; this list belongs in the system description (see reviewer
comments), not in the Results section. See also both reviewer comments regarding
this issue, in both review rounds.}

The tool description has been moved to the ``Implementation'' section which has
been fully re-organised in five sub sections: ``Inputs'', ``Display features'',
``Outputs'', ``Integration'' and ``Web application'' .


\textbf{2. The Discussion section needs to iterate over the advantages of using
jvenn, based on the Results above; then list limitations, assumptions in the
approach, its relation to existing software etc. Mimi Zeiger's "Essentials of
Writing Biomedical Research Papers" is an excellent resource in this respect.}

The ``Results & Discussions'' section has been splited into two distincts
sections. The discussion describes the benefit using jvenn compared to VENNTURE.


\textbf{3. A short biological motivation should be added to the first paragraph
of the Introduction, and also to the abstract; with specific examples of
biological use of Venn diagrams. Many of the BMC Bioinformatics readers are
biologists, and this is the audience that needs to be convinced to use the
software. The current intro and abstract are too generic.}

An introduction with biological examples has been added in the ``Background''
section and in the ``abstract''.


\textbf{4. The language should be further improved (see individual reviewer
comments).}

The paper has been reed and corrected by an english native speaker.



This paper is unpublished and has not been submitted for publication elsewhere. 
We would be grateful if you would consider it for publication in BMC 
bioinformatics.

%----------------------------------------------------------------------------------------

\end{newlfm}
\end{document}