%% BioMed_Central_Tex_Template_v1.06
%%                                      %
%  bmc_article.tex            ver: 1.06 %
%                                       %

%%IMPORTANT: do not delete the first line of this template
%%It must be present to enable the BMC Submission system to
%%recognise this template!!

%%%%%%%%%%%%%%%%%%%%%%%%%%%%%%%%%%%%%%%%%
%%                                     %%
%%  LaTeX template for BioMed Central  %%
%%     journal article submissions     %%
%%                                     %%
%%          <8 June 2012>              %%
%%                                     %%
%%                                     %%
%%%%%%%%%%%%%%%%%%%%%%%%%%%%%%%%%%%%%%%%%


%%%%%%%%%%%%%%%%%%%%%%%%%%%%%%%%%%%%%%%%%%%%%%%%%%%%%%%%%%%%%%%%%%%%%
%%                                                                 %%
%% For instructions on how to fill out this Tex template           %%
%% document please refer to Readme.html and the instructions for   %%
%% authors page on the biomed central website                      %%
%% http://www.biomedcentral.com/info/authors/                      %%
%%                                                                 %%
%% Please do not use \input{...} to include other tex files.       %%
%% Submit your LaTeX manuscript as one .tex document.              %%
%%                                                                 %%
%% All additional figures and files should be attached             %%
%% separately and not embedded in the \TeX\ document itself.       %%
%%                                                                 %%
%% BioMed Central currently use the MikTex distribution of         %%
%% TeX for Windows) of TeX and LaTeX.  This is available from      %%
%% http://www.miktex.org                                           %%
%%                                                                 %%
%%%%%%%%%%%%%%%%%%%%%%%%%%%%%%%%%%%%%%%%%%%%%%%%%%%%%%%%%%%%%%%%%%%%%

%%% additional documentclass options:
%  [doublespacing]
%  [linenumbers]   - put the line numbers on margins

%%% loading packages, author definitions

%\documentclass[twocolumn]{bmcart}% uncomment this for twocolumn layout and comment line below
\documentclass{bmcart}

%%% Load packages
%\usepackage{amsthm,amsmath}
%\RequirePackage{natbib}
%\RequirePackage{hyperref}
\usepackage[utf8]{inputenc} %unicode support
%\usepackage[applemac]{inputenc} %applemac support if unicode package fails
%\usepackage[latin1]{inputenc} %UNIX support if unicode package fails


%%%%%%%%%%%%%%%%%%%%%%%%%%%%%%%%%%%%%%%%%%%%%%%%%
%%                                             %%
%%  If you wish to display your graphics for   %%
%%  your own use using includegraphic or       %%
%%  includegraphics, then comment out the      %%
%%  following two lines of code.               %%
%%  NB: These line *must* be included when     %%
%%  submitting to BMC.                         %%
%%  All figure files must be submitted as      %%
%%  separate graphics through the BMC          %%
%%  submission process, not included in the    %%
%%  submitted article.                         %%
%%                                             %%
%%%%%%%%%%%%%%%%%%%%%%%%%%%%%%%%%%%%%%%%%%%%%%%%%


\def\includegraphic{}
\def\includegraphics{}



%%% Put your definitions there:
\startlocaldefs
\endlocaldefs


%%% Begin ...
\begin{document}

%%% Start of article front matter
\begin{frontmatter}

\begin{fmbox}
\dochead{Research}

%%%%%%%%%%%%%%%%%%%%%%%%%%%%%%%%%%%%%%%%%%%%%%
%%                                          %%
%% Enter the title of your article here     %%
%%                                          %%
%%%%%%%%%%%%%%%%%%%%%%%%%%%%%%%%%%%%%%%%%%%%%%

\title{jvenn: an interactive venn diagram viewer.}

%%%%%%%%%%%%%%%%%%%%%%%%%%%%%%%%%%%%%%%%%%%%%%
%%                                          %%
%% Enter the authors here                   %%
%%                                          %%
%% Specify information, if available,       %%
%% in the form:                             %%
%%   <key>={<id1>,<id2>}                    %%
%%   <key>=                                 %%
%% Comment or delete the keys which are     %%
%% not used. Repeat \author command as much %%
%% as required.                             %%
%%                                          %%
%%%%%%%%%%%%%%%%%%%%%%%%%%%%%%%%%%%%%%%%%%%%%%

\author[
   addressref={aff1},                   % id's of addresses, e.g. {aff1,aff2}
   corref={aff1},                       % id of corresponding address, if any
   noteref={n1},                        % id's of article notes, if any
   email={jane.e.doe@cambridge.co.uk}   % email address
]{\inits{JE}\fnm{Jane E} \snm{Doe}}
\author[
   addressref={aff1,aff2},
   email={john.RS.Smith@cambridge.co.uk}
]{\inits{JRS}\fnm{John RS} \snm{Smith}}

%%%%%%%%%%%%%%%%%%%%%%%%%%%%%%%%%%%%%%%%%%%%%%
%%                                          %%
%% Enter the authors' addresses here        %%
%%                                          %%
%% Repeat \address commands as much as      %%
%% required.                                %%
%%                                          %%
%%%%%%%%%%%%%%%%%%%%%%%%%%%%%%%%%%%%%%%%%%%%%%

\address[id=aff1]{%                           % unique id
  \orgname{Department of Zoology, Cambridge}, % university, etc
  \street{Waterloo Road},                     %
  %\postcode{}                                % post or zip code
  \city{London},                              % city
  \cny{UK}                                    % country
}
\address[id=aff2]{%
  \orgname{Marine Ecology Department, Institute of Marine Sciences Kiel},
  \street{D\"{u}sternbrooker Weg 20},
  \postcode{24105}
  \city{Kiel},
  \cny{Germany}
}

%%%%%%%%%%%%%%%%%%%%%%%%%%%%%%%%%%%%%%%%%%%%%%
%%                                          %%
%% Enter short notes here                   %%
%%                                          %%
%% Short notes will be after addresses      %%
%% on first page.                           %%
%%                                          %%
%%%%%%%%%%%%%%%%%%%%%%%%%%%%%%%%%%%%%%%%%%%%%%

\begin{artnotes}
%\note{Sample of title note}     % note to the article
\note[id=n1]{Equal contributor} % note, connected to author
\end{artnotes}

\end{fmbox}% comment this for two column layout

%%%%%%%%%%%%%%%%%%%%%%%%%%%%%%%%%%%%%%%%%%%%%%
%%                                          %%
%% The Abstract begins here                 %%
%%                                          %%
%% Please refer to the Instructions for     %%
%% authors on http://www.biomedcentral.com  %%
%% and include the section headings         %%
%% accordingly for your article type.       %%
%%                                          %%
%%%%%%%%%%%%%%%%%%%%%%%%%%%%%%%%%%%%%%%%%%%%%%

\begin{abstractbox}

\begin{abstract} % abstract
\parttitle{Background} %if any
In many genomics projects, numerous lists containing biological identifiers are produced. Often 
it is useful to see the overlap between different lists, enabling researchers to quickly observe 
similarities and differences between the data sets they are analyzing. One of the most popular 
methods to visualize the overlap and differences between data sets is the Venn diagram: a diagram 
consisting of two or more circles in which each circle corresponds to a data set, and the overlap 
between the circles corresponds to the overlap between the data sets. Venn diagrams are especially 
useful when they are 'area-proportional' i.e. the sizes of the circles and the overlaps correspond 
to the sizes of the data sets. Currently there are no programs available that can create area-proportional 
Venn diagrams connected to a wide range of biological databases. 

The generation of large volumes of omics data to
conduct exploratory studies has become feasible and is now
extensively used to gain new insights in life sciences. The effective
exploration of the generated data by experts is a crucial step for the
successful extraction of knowledge from these data sets. This
requires availability of intuitive and interactive visualization tools
which can display complex data. Matrix heatmaps are graphical
representations frequently used for the description of complex omics
data. Here we present jHeatmap, a web-based tool which allows
interactive matrix heatmap visualization and exploration. It is an
adaptable javascript library designed to be embedded by means of
basic coding skills into web-portals to visualize data matrices as
interactive and customizable heatmaps.

\parttitle{Results} %if any
We designed a web application named BioVenn to summarize the overlap between two or three lists of 
identifiers, using area-proportional Venn diagrams. The user only needs to input these lists of 
identifiers in the textboxes and push the submit button. Parameters like colors and text size can 
be adjusted easily through the web interface. The position of the text can be adjusted by 'drag-and-drop' 
principle. The output Venn diagram can be shown as an SVG or PNG image embedded in the web application, 
or as a standalone SVG or PNG image. The latter option is useful for batch queries. Besides the Venn diagram, 
BioVenn outputs lists of identifiers for each of the resulting subsets. If an identifier is recognized as 
belonging to one of the supported biological databases, the output is linked to that database. Finally, 
BioVenn can map Affymetrix and EntrezGene identifiers to Ensembl genes. 

\parttitle{Conclusions} %if any
jquery.venny is an easy-to-use web application to generate Venn and Edward diagrams from lists of 
biological identifiers. Its implementation on the World Wide Web makes it available for use on any computer 
with internet connection, independent of operating system and without the need to install programs locally. 
The software package is freely available under the GNU General Public License (GPL) at 
https://mulcyber.toulouse.inra.fr/plugins/mediawiki/wiki/venny/index.php/Accueil. Examples and the documentation 
can be found on the sources directory and a running version is running at http://bioinfo.genotoul.fr/index.php?id=116.

\end{abstract}

%%%%%%%%%%%%%%%%%%%%%%%%%%%%%%%%%%%%%%%%%%%%%%
%%                                          %%
%% The keywords begin here                  %%
%%                                          %%
%% Put each keyword in separate \kwd{}.     %%
%%                                          %%
%%%%%%%%%%%%%%%%%%%%%%%%%%%%%%%%%%%%%%%%%%%%%%

\begin{keyword}
\kwd{venn}
\kwd{edward}
\kwd{diagram}
\kwd{jquery}
\kwd{javascript}
\end{keyword}

% MSC classifications codes, if any
%\begin{keyword}[class=AMS]
%\kwd[Primary ]{}
%\kwd{}
%\kwd[; secondary ]{}
%\end{keyword}

\end{abstractbox}
%
%\end{fmbox}% uncomment this for twcolumn layout

\end{frontmatter}

%%%%%%%%%%%%%%%%%%%%%%%%%%%%%%%%%%%%%%%%%%%%%%
%%                                          %%
%% The Main Body begins here                %%
%%                                          %%
%% Please refer to the instructions for     %%
%% authors on:                              %%
%% http://www.biomedcentral.com/info/authors%%
%% and include the section headings         %%
%% accordingly for your article type.       %%
%%                                          %%
%% See the Results and Discussion section   %%
%% for details on how to create sub-sections%%
%%                                          %%
%% use \cite{...} to cite references        %%
%%  \cite{koon} and                         %%
%%  \cite{oreg,khar,zvai,xjon,schn,pond}    %%
%%  \nocite{smith,marg,hunn,advi,koha,mouse}%%
%%                                          %%
%%%%%%%%%%%%%%%%%%%%%%%%%%%%%%%%%%%%%%%%%%%%%%

%%%%%%%%%%%%%%%%%%%%%%%%% start of article main body
% <put your article body there>

%%%%%%%%%%%%%%%%
%% Background %%
%%

\section*{Background}
In many genomics projects and other projects handling large amounts of biological data, various lists 
containing biological identifiers are produced, corresponding to e.g. sets of genes regulated under different 
treatments. Often, it is useful to see the overlap between these lists. This enables researchers to quickly 
observe similarities and differences between the data sets they are analyzing. One of the most popular methods 
to visualize the overlap and differences between data sets is the Venn diagram, named by its inventor John Venn 
\cite{Venn1880}. A large number of different types of Venn diagrams exist, the most popular being the three-circle 
Venn diagram, used to visualize the overlap between three data sets. In such a diagram, the size of the circle can 
be used to represent the size of the corresponding data set. This is called an area-proportional Venn diagram 
[2]. Venn diagrams have been used recently to visualize gene lists [3,4]. However, these applications generate 
diagrams with circles of equal size. 

There are some computer programs available that generate area-proportional Venn Diagrams, either as rectangles 
[5] or as polygons [6]. Drawback of these programs is that they need to be downloaded and run locally, limiting 
their use by a wide community. There is also the Google Chart API [7], which can generate circular, area-proportional 
Venn Diagrams, but can only have three numbers as input, and cannot do any calculations to obtain these three numbers. 
There is currently no web application available that can generate circular, area-proportional Venn diagrams connected 
to a wide range of biological databases, and can map different kinds of IDs to genes. In this article, we present a 
web application named BioVenn which can generate circular, area-proportional Venn diagrams just by entering two or 
three lists of biological IDs. IDs that can be recognized by BioVenn as belonging to a certain database, are linked 
to that database. BioVenn currently supports cross-references to Affymetrix [8], COG [9], Ensembl [10], EntrezGene [11], 
Gene Ontology [12], InterPro [13], IPI [14], KEGG Pathway [15], KOG [16], PhyloPat [17] and RefSeq [18]. BioVenn is based 
on a previous version [19], which has been used in several scientific publications to visualize sets and their overlapping 
areas [20-22]. 

The need for effective tools for data visualization is rising with the
increasing data volumes generated by scientific studies. Effective
data visualization lets the researcher understand his/her data at
both broad and detailed levels and enables barrier-free exploration
of the data sets. A widely used type of visualization to report
biological results are matrix heatmaps, that represent a data set
with two dimensions, commonly genes and samples. The values in
the matrix may represent any widely measurable property such as
expression values. The static nature of such plots is a limiting
factor in order to explore complex data sets. Therefore we
introduced the use of interactive heatmaps and developed Gitools,
a desktop application for this purpose (Perez-Llamas and Lopez-
Bigas, 2011). With jHeatmap we provide a javascript library which
can be included in any web-platform to interactively explore
heatmaps over a web browser without any further software
barriers. This could be compared to the creation of Cytoscape-Web
(Lopes et al., 2010) which maintains basic features for
network visualization of Cytoscape desktop application. The
data is visualized in a heatmap matrix that can contain multiple
values per cell and thus allows loading multidimensional data sets
such as alteration data from oncogenomic study cohorts. Each cell
is associated to two features, e.g. genes and samples, with
additional information, e.g. clinical information for samples, and a
set of values. Columns and rows can be moved freely, and can be
filtered and sorted based on values in the cells or based on
annotations. All of properties associated to rows and columns can
be color coded or printed as text.

\section*{Implementation}
a jquery plugin 
all figures are a canvas object 
The jquery.venny main features are: up to 6 classes venn diagram, allow to display Edwards-Venn diagram, easily integrable 
within your own web site, 3 different ways to provide the data (list/number/list+number), control the click callback function,
export the venn diagram to PNG.

jHeatmap is a JavaScript plugin for jQuery and only has to be
referenced from within a HTML file with the necessary
configuration linking the data matrix file and the visualization
options. The browser will load the data and draw the heatmap in
the web browser as desired.
As depicted in Figure 1, the heatmap has four components: The
matrix heatmap, labels and color annotations for columns on top
and for rows to the left and right and finally a control area on the
top left. In the control area the user may select what value is
displayed in case the cells, rows or columns are associated with
multiple values. For each of the three areas, a drop-down menu
reveals to the user the different values that are available for cells,
rows or columns respectively. Columns and rows can be selected,
moved, sorted and hidden which gives the user great flexibility to
focus on the data points of interest. Columns and rows can be
selected by click and drag actions. Clicking on ♦ will sort the
matrix by the selected rows or columns. Rows and columns used
for sorting will be labeled with triangles (▲ ▼) indicating the
ordering applied, and clicking them will toggle the direction. The
row and column annotations offer the same possibility. Further
interactions can be accessed through a contextual menu revealed
by a long click on the row or column labels.
jHeatmap is designed for incorporation into web portals and
applications. Documentation about several extension points can be
found at the jHeatmap website.

\section*{Results and Discussion}

Given the possibility to generate large amount of biological data
with high-throughput technologies the need for data visualization
and analysis in biology is increasing. The shift from hypothesisdriven
to data-driven analyses requires field experts to directly
access the data. The visual and interactive access to complex data
enables experts to reason and decide over further analytical
procedures. For example in cancer genomics researchers need to
visualize and analyze complex multidimensional genomic data
often of large number of patients (Schroeder et al., 2013). For the
above stated reasons we present the jHeatmap software library for
easy representation of big data sets. jHeatmap is already in use
within the IntOGen platform (Gundem et al., 2010; Gonzalez-
Perez et al., 2013), the Achilles Project (Cheung et al., 2011) and
GenomeSpace (Liefeld, 2013) web platforms. jHeatmap is open
source, reusable and extendable: the web creator may easily
include the library to existing projects in order to visualize
multidimensional data sets of any size. Basic coding skills are
required. For more advanced users it is possible to adapt and
extend the code as needed.
With jHeatmap, we complement already existing stand-alone
desktop solutions for interactive heatmaps such as Gitools (Perez-
Llamas and Lopez-Bigas, 2011) or Java TreeView (Saldanha,
2004) with an open source matrix heatmap visualizer prepared for
integration into webs. jHeatmap offers a set of actions to provide
an interactivity that allows the user to visually mine the data:
According to the defined software interactions (Yi et al., 2007)
jHeatmap lets the user select, explore, reconfigure, encode and
filter rows, columns and cells of the heatmap.


\subsection*{How to embed jquery.venny in your own web application}

\subsection*{How to use our web application}

\section*{Conclusions}

\section*{Availability and requirements}

%%%%%%%%%%%%%%%%%%%%%%%%%%%%%%%%%%%%%%%%%%%%%%
%%                                          %%
%% Backmatter begins here                   %%
%%                                          %%
%%%%%%%%%%%%%%%%%%%%%%%%%%%%%%%%%%%%%%%%%%%%%%

\begin{backmatter}

\section*{Competing interests}
  The authors declare that they have no competing interests.

\section*{Author's contributions}
    Text for this section \ldots

\section*{Acknowledgements}
We would like to acknowledge all our users for providing us useful feedback on
the system and for pointing out features worth developing.

%%%%%%%%%%%%%%%%%%%%%%%%%%%%%%%%%%%%%%%%%%%%%%%%%%%%%%%%%%%%%
%%                  The Bibliography                       %%
%%                                                         %%
%%  Bmc_mathpys.bst  will be used to                       %%
%%  create a .BBL file for submission.                     %%
%%  After submission of the .TEX file,                     %%
%%  you will be prompted to submit your .BBL file.         %%
%%                                                         %%
%%                                                         %%
%%  Note that the displayed Bibliography will not          %%
%%  necessarily be rendered by Latex exactly as specified  %%
%%  in the online Instructions for Authors.                %%
%%                                                         %%
%%%%%%%%%%%%%%%%%%%%%%%%%%%%%%%%%%%%%%%%%%%%%%%%%%%%%%%%%%%%%

% if your bibliography is in bibtex format, use those commands:
\bibliographystyle{bmc-mathphys} % Style BST file
\bibliography{bmc_venny}      % Bibliography file (usually '*.bib' )

% or include bibliography directly:
% \begin{thebibliography}
% \bibitem{b1}
% \end{thebibliography}

%%%%%%%%%%%%%%%%%%%%%%%%%%%%%%%%%%%
%%                               %%
%% Figures                       %%
%%                               %%
%% NB: this is for captions and  %%
%% Titles. All graphics must be  %%
%% submitted separately and NOT  %%
%% included in the Tex document  %%
%%                               %%
%%%%%%%%%%%%%%%%%%%%%%%%%%%%%%%%%%%

%%
%% Do not use \listoffigures as most will included as separate files

\section*{Figures}
  \begin{figure}[h!]
  \caption{\csentence{Sample figure title.}
      A short description of the figure content
      should go here.}
      \end{figure}

\begin{figure}[h!]
  \caption{\csentence{Sample figure title.}
      Figure legend text.}
      \end{figure}

\end{backmatter}
\end{document}
